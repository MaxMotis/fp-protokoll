












%\usepackage{picins}




%------ math stuff --------------
\usepackage{amsfonts} % für verschiedene Fonts, z.b. Blackbaord Bold
\usepackage{amsmath}
\usepackage{amssymb}
\usepackage{amsthm}
%\usepackage{mathtools} %mehr Kontrolle über (das statische Paket) amsmath
\usepackage{unicode-math} %typesetting von verschiedenen Symbolen, insbesondere aufrechte griechische Buchstaben
%\usepackage{bm} % kann Math-Symbole fett machen


%------- counter ----------------
\usepackage{chngcntr} %Zum Ändern der Nummerierung von Equations und so
\counterwithin{figure}{section}
\counterwithin{table}{section}
%\numberwithin{equation}{subsection}


%-------- physics typesetting ----------
\usepackage{physics}
%\usepackage[arrowdel]{physics} %Verwende das für Nabla mit Vektorpfeil
%\usepackage{esvect} %für bessere und verschiedene Vektorpfeile
%\usepackage{siunitx} %für den Umgang mit Einheiten und so
%\sisetup{locale = DE}


%-------- general document stuff ----------
\usepackage{import} %falls eine importierte Datei weitere \input{}s hat
%\usepackage{multicol} %um Text in mehreren Spalten aufzureihen
%\usepackage{xcolor} %für besseres Farben-Management
\usepackage{hologo} % Logos für verschiedene \Latex Varianten
\usepackage[english, main=ngerman]{babel}


%------- eigene control sequences --------------
\newcommand{\comment}[1]{} % Verwendet, um ganze Textblöcke auszukommentieren
\renewcommand{\r}[1]{\mathrm{#1}} % handliche Version von \mathrm

%-------- citation crustacean  -----------------
\usepackage[backend=biber,sorting = none, style=phys]{biblatex}
\usepackage{hyperref} %verschiedene Möglichkeiten, Referencen und on-click jumps einzufügen
\usepackage[autostyle]{csquotes}

%--------- tabel stuff -----------------
\usepackage{booktabs} %viel bessere Tabellen (visuell)
\usepackage{multirow} %für Tabellen und Aufreihung mit großen Klammern zur Verbindung
\usepackage{tabularx} %soll ein besseres Tabellenenvironment sein

%----------- allgemeiner float Kram -------------
\usepackage{subcaption} %captions für subfigures
%\usepackage{float} %mehr float Optionen
%\usepackage{floatflt} %mehr float Optionen
\usepackage{graphicx} %Grafiken einfügen u.v.m
%\usepackage{caption} %für caption-control von floating envs, optionen z.B. [margin=10pt,font=small,labelfont=bf,labelsep=endash]